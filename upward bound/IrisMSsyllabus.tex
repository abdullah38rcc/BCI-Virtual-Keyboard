\documentclass[11pt]{article}
\usepackage[margin=1in]{geometry}
\usepackage{fancyhdr}
\usepackage{multicol}
\usepackage{amsmath}
\usepackage{amssymb}
\usepackage{cancel}
\usepackage{hyperref}
\usepackage{url}

\pagestyle{fancy}
\lhead{Summer 2010}
\chead{Calculus}
\rhead{McLeary}

\newcommand{\ds}{\displaystyle}
\newcommand{\degree}{\ensuremath{^\circ}}
\DeclareMathOperator*{\sgn}{sgn}
\newcommand{\Ints}{\mathbb{Z}}
\newcommand{\Real}{\mathbb{R}}
\newcommand{\Naturals}{\mathbb{N}}


\begin{document}

\title{Course Information \& Syllabus}
\author{Iris McLeary}
\date{}
\maketitle
\thispagestyle{fancy}


\section{Course Details}

\begin{description}
\item[Instructor:] Iris McLeary, \href{mailto:iris.mcleary@gmail.com}{iris.mcleary@gmail.com}
\item[Teaching Assistant:] Ron Page, \href{mailto:xlpage@gmail.com}{xlpage@gmail.com}
\end{description}


\section{Course Description}

Like geometry is about shapes, calculus is about \emph{rates of change}---specifically, how one variable changes in relation to another. Calculus is essential in fields ranging from engineering to medicine to economics. In this course, we will begin with a quick review of precalculus fundamentals, then move on to limits and differential calculus. The goal of this class is to prepare you to take AP-level calculus in the fall and expose you to the real-world applications of the math we are studying.


\section{Course Reader}

All readings for this course are in your course reader, which you should bring to class with you every day. Your reader is built from many sources, all of which are listed immediately after the table of contents. Part of your homework every night is to read the assigned section(s) for the next day in order to be prepared for class.


\section{Course Website}

There is a course website at \href{http://www-inst.eecs.berkeley.edu/~ias/ub/ms_calc1/}{\url{http://www-inst.eecs.berkeley.edu/~ias/ub/ms_calc1/}}, which will have links to all homework assignments, any handouts distributed in class, and the lecture notes.


\section{Requirements \& Grading Breakdown}

\begin{itemize}
\item Notebook containing class notes and a glossary of terms with definitions developed over the course of the summer: 10\%.
\item Nightly homework covering \emph{that day's} material: 20\%.
\item Weekly quizzes (at the beginning of each week): 15\%.
\item One midterm exam on July 9\textsuperscript{th} (covering material through July 2\textsuperscript{nd}): 15\%.
\item Final exam (July 21\textsuperscript{st}): 25\%.
\item Class and study group attendance \& participation: 15\%.
\end{itemize}


\section{Notebook}

Part of this class is to teach you good study habits and note-taking skills. Your notebook will be turned in the day of the final and will be graded on organization and completeness. At the back of the notebook, you will maintain a glossary of terms with definitions drawn from nightly reading and lectures. \emph{Any word or mathematical term you come across and don't know} should go in the glossary.


\section{Homework}

Homework assignments will be distributed every day in class and will be due the following day at the beginning of class. The only exceptions are the days of the midterm and final exams---the previous day's homework will be due the day \emph{after} the exam to allow you time to study. Late homework will be penalized 10\% per day, and will trigger a conference with me and possible further consequences.


\section{Quizzes}

At the beginning of the first class period of every week, we will have a 10-minute quiz on the previous week's material. \emph{If you are late to class}, you will receive a zero on the quiz.


\section{Exams}

There will be one midterm exam and a final exam. For both exams, you will be permitted to have \emph{one} $8.5 \times 11$ sheet of paper, double-sided, with any notes and formulas you like. This cheat sheet will be turned in with your exam. Making a cheat sheet is one of the best ways to study for an exam, and 5\% of your exam grade will be based on your cheat sheet.


\section{Tutoring \& Group Work}

Your TA (Ron Page) will be available Monday through Thursday evenings during study hours to assist you with your homework. Group study sessions led by me will be held on Tuesday or Wednesday evenings. When a study session is scheduled, you will be notified in class at least 24 hours in advance. All group study sessions are \emph{mandatory}.


\section{Revisions to the Syllabus}

This syllabus may be revised depending on the pace at which we cover the material. If any revisions are made, you will receive a new syllabus the next day in class.


\section{Course Calendar}

\begin{center}
\begin{tabular}{|c|c|c|c|}
\hline
\textbf{Date} & \textbf{Topic} & \textbf{Reading} & \textbf{Due} \\
\hline \hline
6/18 & Course Intro; Real Numbers, Sets, Intervals, \& Inequalities & \S~1--2 & --- \\
\hline
6/21 & Absolute Values; Coordinate Planes and Lines & \S~3--4 & HW 1 \\
\hline
6/22 & Distance, Circles, and Quadratic Equations & \S~5 & HW 2 \\
\hline
6/23 & Physics: One-dimensional Motion with Constant Acceleration & \S~6 & HW 3 \\
\hline
6/24 & Trigonometry Review & \S~7 & HW 4 \\
\hline
6/25 & Physics: Reflection and Refraction & \S~8 & HW 5 \\
\hline
6/28 & Functions; Properties of Functions & \S~9--10 & HW 6 \\
\hline
6/29 & Transformation of Functions & \S~11 & HW 7 \\
\hline
6/30 & Families of Functions; Logarithmic and Exponential Functions & \S~12--13 & HW 8 \\
\hline
7/1 & Introduction to Limits, Asymptotes, and Continuity & \S~14 & HW 9 \\
\hline
7/2 & Calculating Limits & \S~15 & HW 10 \\
\hline
7/6 & Formal Definition of Limits & \S~16 & HW 11 \\
\hline
7/7 & Continuity & \S~17 & HW 12 \\
\hline
7/8 & \textbf{MIDTERM EXAM (material through 7/6)} & --- & --- \\
\hline
7/11 & Tangent Lines and Rates of Change & \S~18 & HW 13 \\
\hline
7/12 & The Derivative & \S~19 & HW 14 \\
\hline
7/13 & The Power Rule and Linearity & \S~20 & HW 15 \\
\hline
7/14 & The Product and Quotient Rules & \S~21 & HW 16 \\
\hline
7/15 & Derivatives of Trigonometric Functions & \S~22 & HW 17 \\
\hline
7/16 & The Chain Rule & \S~23 & HW 18 \\
\hline
7/19 & Local Linear Approximation and Differentials & \S~24 & HW 19 \\
\hline
7/20 & Maxima and Minima; The First Derivative Test & \S~25 & HW 20 \\
\hline
7/21 & \textbf{FINAL EXAM} & --- & --- \\
\hline
7/22 & \textbf{POST EXAM} & --- & HW 21 \\
\hline
\end{tabular}
\end{center}


\end{document}